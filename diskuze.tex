\section*{Diskuze}
V grafech \ref{g:h} a \ref{g:k} jsme závislost fitovali funkcí tvaru \cite{skripta}
\begin{equation*}
n(\lambda) = n_0 + \frac{a}{\lambda + \lambda_0} \,.
\end{equation*}
Pro hranol jsme použili parametry
\begin{equation*}
n_0=1,49 \qquad \qquad a=\SI{10.4}{\per\nm} \qquad \qquad \lambda_0=\SI{-108}{\nm}
\end{equation*}
pro kapalinu pak
\begin{equation*}
n_0=1,32 \qquad \qquad a=\SI{6.6}{\per\nm} \qquad \qquad \lambda_0=\SI{-130}{\nm} \,.
\end{equation*}
Podle grafu tato funkce závislost dobře aproximuje.


Nejsme si vědomi žádné systematické chyby, které bychom se dopustili. Systematická chyba způsobená nepřesným odečítáním ze stupnice goniometru působila při měření ve druhém směru opačně, takže se vyrušila.

Měření považujeme za velice přesné a hodnotami indexů lomů jsme si skutečně jisti na tři číslice za desetinnou čárkou.

Při výpočtu střední disperze, relativní disperze a Abbeova čísla se odečítají blízké hodnoty, což má za následek vysokou relativní chybu. Díky přesnosti měření indexu lomu je však i tato chyba rozumná.

